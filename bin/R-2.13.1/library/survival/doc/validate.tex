\documentclass[11pt]{article}
\title {Validation for the Cox Model}
\author{Terry M. Therneau \\
        Mayo Foundation }
\date{Oct 2010}

\def\bhat{\hat \beta}        %define "bhat" to mean "beta hat"
\def\Mhat{\widehat M}        %define "Mhat" to mean M-hat
\def\xbar{\bar x}
\def\lhat{\hat \Lambda}
\def\Ybar{\overline{Y}}
\def\Nbar{\overline{N}}
\def\Vbar{\overline{V}}
\def\yhat{\hat y}
\def\imat{{\cal I}}
\def\Cvar{{\cal I}^{-1}}

\begin{document}
\maketitle


It is useful to have a set of test data where the results have been
worked out in detail, both to illuminate the computations and to form
a test case for software programs.
The data sets below are quite simple, but have proven useful in this
regard.


\section{Test data 1}
In this data set $x$ is a 0/1 treatment covariate, with $n=6$ subjects. 
There is one tied death time, one time with a death and a censoring,
one with only a death, and one with only a censoring.
(This is as small as a data set can be and still cover the four cases.) 
Let $r = \exp(\beta)$ be the risk score for a subject with $x=1$.
Table \ref{tab:val1} shows the data set along with the mean and
increment to the hazard at each point.

\begin{table} \centering
\begin{tabular}{ccc|cc|cc}
&&& \multicolumn{2}{c|}{$\xbar(t)$} & \multicolumn{2}{c}{$d\lhat_0(t)$} \\
Time&Status&$x$&Breslow&Efron&Breslow&Efron \\ \hline 
1&1&1&$r/(r+1)$ & $r/(r+1)$& $1/(3r+3)$ & $1/(3r+3)$ \\
1&0&1&&&&\\
6&1&1& $r/(r+3)$ & $r/(r+3)$ & $2/(r+3)$& $1/(r+3)$ \\
6&1&0& &  $r/(r+5)$ && $2/(r+5)$ \\
8&0&0&&&& \\
9&1&0&0&0& 1& 1\\
\end{tabular}
\caption{Test data 1}
\label{tab:val1}
\end{table}

\subsection{Breslow estimates}
\label{sect:valbreslow}
The loglikelihood has a term for each event; each term is
the log of the ratio of the score for the subject who had an event over
the sum of scores for those who did not.
\begin{eqnarray*}
LL &=& \{\beta- \log(3r+3)\} + \{\beta - \log(r+3)\} + \{0-\log(r+3)\} + \{0-0\} \\
   &=& 2\beta - \log(3r+3) - 2\log(r+3). \\ \\
U &=& \left(1-\frac{r}{r+1}\right) + \left(1-\frac{r}{r+3}\right) + 
	\left(0-\frac{r}{r+3} \right) + (0-0) \\
  &=& \frac{-r^2 + 3r + 6}{(r+1)(r+3)} .\\ \\
\imat&=&     \left\{\frac{r}{r+1} - \left( \frac{r}{r+1} \right )^2\right\}
        +2 \left\{\frac{r}{r+3} - \left( \frac{r}{r+3} \right )^2\right\} +
		 (0-0) \\
  &=& \frac{r}{(r+1)^2} + \frac{6r}{(r+3)^2}.
\end{eqnarray*}

The actual solution corresponds to $U(\beta)=0$, which from the quadratic
formula is $r=(1/2)(3 + \sqrt{33}) \approx 4.372281$,
or $\bhat = \log(r) \approx 1.475285$.  
Then
$$
\begin{array}{ll}
 LL(0)=-4.564348 & LL(\bhat)=-3.824750 \\
 U(0) = 1 & U(\bhat) = 0 \\
 \imat(0) = 5/8 =0.625 & \imat(\bhat) = 0.634168 \,.
\end{array}
$$
Newton--Raphson iteration has increments of $-\Cvar U$.
Starting with the usual initial estimate of $\beta=0$, the
N--R iterates are $0$, $8/5$, $1.4727235$, $1.4752838$, $1.4752849$, \ldots .
S considers the algorithm to have converged after three iterations,
SAS after four (using the default convergence criteria in each package).

The martingale residuals are a simple function of the cumulative hazard,
$M_i = \delta_i - r\lhat(t_i)$.
\begin{center}
\begin{tabular}{c|lrr}
Subject &\multicolumn{1}{c}{$\Lambda_0$} & $\Mhat(0)$ & $\Mhat(\bhat)$ \\ 
	\hline
1& $1/(3r+3)$ & 5/6 & .728714 \\
2& $1/(3r+3)$ & --1/6& --.271286 \\
3& $1/(3r+3) + 2/(r+3)$ &1/3 & --.457427 \\
4& $1/(3r+3) + 2/(r+3)$ &1/3 & .666667 \\
5&   $1/(3r+3) + 2/(r+3)$ &--2/3 & --.333333 \\
6 &   $1/(3r+3) + 2/(r+3) +1$ & --2/3 & --.333333 
\end{tabular}
\end{center}

The score residual $L_i$ can be calculated from the entries in
 Table \ref{tab:val1}.
For subject number 3, for instance, we have
\begin{eqnarray*}
 L_3 &=&\int_0^6 \{1 - \xbar(t)\}  d\Mhat_3(t) \\
	&=& \left(1-\frac{r}{r+1}\right) \frac{r}{3r+3} + 
	\left(1-\frac{r}{r+3}\right) \left(1-\frac{2r}{r+3}\right) .
\end{eqnarray*}
Let $a=(r+1)(3r+3)$ and $b=(r+3)^2$; then the residuals are as follows.
\begin{center}
\begin{tabular}{c|crr}
Subject & $L$& $L(0)$ &$ L(\bhat)$ \\ \hline
1& $(2r+3)/a$& 5/12 & .135643 \\
2& $-r/a$& --1/12& --.050497 \\
3& $-r/a + 3(3-r)/b$& 7/24& --.126244 \\
4& $r/a - r(r+1)/b$& --1/24& --.381681\\
5& $r/a + 2r/b$& 5/24& .211389 \\
6& $r/a + 2r/b$& 5/24& .211389 
\end{tabular}
\end{center}

The Schoenfeld residuals are defined at the three unique death times, and
have values of $1-r/(r+1) = 1/(r+1)$, 
$\{1 -r/(r+3)\} + \{0- r/(r+3)\}= (3-r)/(3+r)$,
and 0 at times 1, 6, and 9, respectively.
For convenience in plotting and use, however, the programs return one
residual for each event rather than one per unique event time.
The two values returned for time 6 are $3/(r+3)$ and $-r/(r+3)$.

The Nelson--Aalen estimate of the hazard is closely related to the
Breslow approximation for ties.
The baseline hazard is shown as the column $\Lambda_0$ above. 
The hazard estimate for a subject with covariate $x_i$ is
$\Lambda_i(t) = \exp(x_i \beta) \Lambda_0(t)$ and the survival
estimate is $S_i(t)= \exp(-\Lambda_i(t))$.

    The variance of the cumulative hazard is the sum of two terms.  
Term 1 
is a natural extension of 
the Nelson--Aalen estimator to the case where there are
weights.  It is a running sum, with an increment at each death time of
$dN(t)/(\sum Y_i(t)r_i(t))^2$.
For a subject with covariate $x_i$ this term is multiplied by
$[\exp(x_i \beta)]^2$.

    The second term is $d\Cvar d'$, where $\imat$ is the 
variance--covariance matrix of the
Cox model, and $d$ is a vector.  
The second term accounts for the fact that the
weights themselves have a variance; 
$d$ is the derivative of $S(t)$ with respect to $\beta$
and can be formally written as
$$         \exp(x\beta) \int_0^t (\bar x(s) - x_i) d\hat\Lambda_0 (s)\,.
$$
This can be recognized as $-1$ times the score residual process
for a subject with $x_i$ as covariates and no events; it measures
leverage of a particular observation on the estimate of $\beta$.  It is
intuitive that a small score residual --- an obs with such 
covariates has little influence on $\beta$ --- results in a small added
variance; that is, $\beta$ has little influence on the estimated survival.  

\begin{center}
\begin{tabular}{c|l}
Time & Term 1  \\ \hline
1& $1/(3r+3)^2$ \\
6& $1/(3r+3)^2 + 2/(r+3)^2$ \\
9& $1/(3r+3)^2 + 2/(r+3)^2 + 1/1^2$ \\ \multicolumn{2}{c}{} \\
Time  & $d$ \\ \hline
1&  $(r/(r+1))* 1/(3r+3)$ \\
6&  $(r/(r+1))* 1/(3r+3) + (r/(r+3))* 2/(r+3)$ \\
9& $(r/(r+1))* 1/(3r+3) + (r/(r+3))* 2/(r+3)  + 0*1$\\
\end{tabular}
\end{center}

 For $\beta=0$, $x=0$:
\begin{center}
\begin{tabular}{c|lll}
     Time  &   \multicolumn{2}{c}{Variance}&\\ \hline 
       1   &      1/36            &+ $1.6*(1/12)^2          $ &= 7/180\\
       6   &     (1/36 + 2/16)    &+ $1.6*(1/12 + 2/16)^2   $ &= 2/9\\
       9   &     (1/36 + 2/16 + 1)&+ $1.6*(1/12 + 2/16 + 0)^2$&= 11/9\\
\end{tabular}
\end{center}

   For $\beta=1.4752849$, $ x=0$
\begin{center}
\begin{tabular}{c|lll}
     Time  &   \multicolumn{2}{c}{Variance}&\\ \hline 
       1  &     0.0038498  &+  .004021  &= 0.007871\\
       2   &    0.040648   &+ .0704631  &= 0.111111\\
       4    &   1.040648   &+ .0704631  &= 1.111111\\
\end{tabular}
\end{center}



\subsection{Efron approximation}
The Efron approximation \cite{Efron77}
differs from the Breslow only at day 6, where two
deaths occur.  
A useful way to think about the approximation is this: assume that if the
data had been measured with higher accuracy that the deaths would not have been
tied, that is two cases died on day 6 but they did not perish at the
same instant on that day.  
There are thus two separate events on day 6.  Four subjects were alive and
at risk for the first of the events.  Three subjects were at risk for the
second event, either subjects 3, 5, and 6 or subjects 2, 5, and 6, but we
do not know which.  
In some sense then, subjects 3 and 4 each have ``.5'' probability of 
being at risk for the
second event at time $2 + \epsilon$.
In the computation, we treat the two deaths as two separate times (two terms
in the loglik), with subjects 3 and 4 each having a case weight of 1/2 for the
second event.
The mean covariate for the second event is then
$$
 \frac{1*r/2 + 0*1/2 + 0*1 + 0*1 }
      {r/2 + 1/2 + 1+1} =  \frac{r}{r+5}
$$
and the main quantities are
\begin{eqnarray*}
LL &=& \{\beta- \log(3r+3)\} + \{\beta - \log(r+3)\} + \{0-\log(r/2 +5/2)\} 
	+ \{0-0\}  \\
   &=& 2\beta - \log(3r+3) - \log(r+3) - \log(r/2 +5/2)\\ \\
U &=& \left(1-\frac{r}{r+1}\right) + \left(1-\frac{r}{r+3}\right) 
	+ \left(0-\frac{r}{r+5}\right) + (0-0) \\
  &=& \frac{-r^3 + 23r + 30}{(r+1)(r+3)(r+5)} \\ \\
I&=&     \left\{\frac{r}{r+1} - \left( \frac{r}{r+1} \right )^2\right\}
        + \left\{\frac{r}{r+3} - \left( \frac{r}{r+3} \right )^2\right\} \\
  &&    + \left\{\frac{r}{r+5} - \left( \frac{r}{r+5} \right )^2\right\}. 
\end{eqnarray*}

The solution corresponds to the one positive root of $U(\beta)=0$, which 
can be written as $\phi=\arccos\{(45/23)\sqrt{3/23}\}$, 
$r=2\sqrt{23/3}\cos(\phi/3) \approx 5.348721$,
or $\bhat = \log(r) \approx 1.676858$.
%\emph{Query- this formula is from the CRC handbook Standard Mathematical
%Tables.  Do I need to cite it?}
  
Then
$$
\begin{array}{ll}
 LL(0)=-4.276666 & LL(\bhat)=-3.358979 \\
 U(0) = 52/48 & U(\bhat) = 0 \\
 \imat(0) = 83/144  & \imat(\bhat) = 0.652077.
\end{array}
$$

The cumulative hazard now has a jump of size
$1/(r+3) + 2/(r+5)$ at time 6.
Efron \cite{Efron77} did not discuss estimation of the cumulative
hazard, 
but it follows directly from the same argument as that used for
the loglikelihood so we refer to it as the ``Efron'' estimate of
the hazard.
In S this hazard is the default whenever the Efron approximation for
ties is used; the estimate is not available in SAS.
For simple survival curves (i.e., the no-covariate case), the estimate is
explored by Fleming and Harrington \cite{Fleming84}
as an alternative to the Kaplan--Meier.

  The variance formula for the baseline hazard function
is extended in the same way, and is
the sum of $(\mbox{hazard increment}) ^2$, 
treating a tied death as $d$ separate
hazard increments.
In term 1 of the variance, the increment at time 6 is
now $1/(r+3)^2   + 4/(r+5)^2$ rather than $2/(r+3)^2$.
The increment to $d$ at time 6 is
$(r/(r+3))* 1/(r+3) + (r/(r+5))* 2/(r+5)$.
(Numerically, the result of this
computation is intermediate between the 
Nelson--Aalen variance and the
Greenwood variance used in the Kaplan--Meier, which is
an increment of
$$  \frac{dN(t)}{[\sum Y_i(t)r_i(t)]
	         [\sum Y_i(t)r_i(t) - \sum dN_i(t) Y_i(t)r_i(t)]}\,.$$
The denominator for the Greenwood formula 
is the sum over those at risk, times that sum \emph{without}
the deaths.  
At time 6 this latter is $2/[(r+3)(3)]$.)

 For $\beta=0$, $x=0$, let $v= \Cvar = 144/83$.
\begin{center}
\begin{tabular}{c|ll}
     Time  &   \multicolumn{2}{c}{Variance}\\ \hline 
       1   &      1/36          \\  &
	\quad + $v(1/12)^2  $ &= \phantom{0}119/2988\\
       6   &     (1/36 + 1/16 + 4/25)& \\ 
	         &\quad + $v(1/12 + 1/16+ 1/18)^2$ &= 1996/6225\\
       9   &     (1/36 + 1/16 + 4/25 + 1) \\&
	\quad + $v(1/12 + 1/16 + 1/18 +0)^2$&= 8221/6225\\
\end{tabular}
\end{center}

   For $\beta=1.676857$, $ x=0$.
\begin{center}
\begin{tabular}{c|ll}
     Time  &   \multicolumn{2}{c}{Variance}\\ \hline 
       1  &     0.00275667 + .00319386 & = 0.0059505\\
       2  &     0.05445330 + .0796212  & = 0.134075\\
       4  &     1.05445330 + .0796212   &= 1.134075\\
\end{tabular}
\end{center}

Given the cumulative hazard, the martingale and score residuals
follow directly using similar computations.
Subject 3, for instance, experiences a total hazard of
$1/(3r+3)$ at the first death time, $1/(r+3)$ at the ``first'' death on
day 6, and $(1/2)*2/(r+5)$ at the ``second'' death on day 6
--- notice the case weight of 1/2 on the last term.
Subjects 5 and 6 experience the full hazard of $1/(r+3) + 2/(r+5)$ 
on day 6.
The values of the 
martingale residuals are as follows.
\begin{center}
\begin{tabular}{c|rr}
Subject& $\Mhat(0)$ & $\Mhat(\bhat)$\\ \hline
1&  5/6 & .719171 \\
2&  --1/6& --.280829 \\
3& 5/12 & --.438341  \\
4& 5/12 & .731087  \\
5& --3/4 & --.365543 \\
6 &--3/4 & --.365543 \\
\end{tabular}
\end{center}
Let $a= r+1$, $b= r+3$, and $c=r+5$; then the score residuals are
\begin{center}
\begin{tabular}{c|crr}
Subject& Score& $L(0)$ & $L(\bhat)$ \\ \hline
1& $2b/3a^2$& 5/12 & .113278 \\
2&  $-r/3a^2$& -1/12& --.044234 \\
3&  $\-1/3a^2 + a/2b^2 + b/2c^2$&
		 55/144& --.102920 \\
4&  $r(1/3a^2 - a/2b^2 - b/2c^2$
		&--5/144& --.407840\\
5&  $\frac{2r(177 + 282r + 182r^2 + 50r^3 + 5r^4)}{3a^2 b^2 c^2}$
 		& 29/144&  .220858\\
6 & same & 29/144& .220858
\end{tabular}
\end{center}

For subject 3, the score residual was computed as
$$
\begin{array}{rl}
\left(1-\frac{r}{r+1}\right)\left(0-\frac{1}{3r+3}\right) & +
\left(1-\frac{r}{r+3}\right)\left(\frac{1}{2}-\frac{1}{r+3}\right)\\ & +
\left(1-\frac{r}{r+5}\right)\left(\frac{1}{2}-\frac{1}{r+5}\right) \,;
\end{array}
$$
the single death is counted as 1/2 event for each of the two day 6 events.
Another equivalent approach is to actually form a second
data set in which subjects 3 and 4 are each represented by
two observations, one at time 6 and the other at time $6+\epsilon$,
each with a case weight of 1/2.
Then a computation using the Breslow approximation will give
this score residual as the weighted sum of the score residuals
for the two psuedo-observations.

The Schoenfeld residuals for the first and last events are identical to
the Breslow estimates, that is, $1/(r+1)$ and 0, respectively.
The residuals for time 6 are $1-c$ and $0-c$, where
$c= (1/2)\{r/(r+3) + r/(r+5)\}$, the ``average'' $\xbar$ over the deaths.

It is quite possible to combine the Efron approximation for $\bhat$
along with the Breslow (or Nelson--Aalen) estimate of $\lhat$,
and in fact this is the behavior used in some packages.
That is, if the \texttt{ties=efron} option is chosen the formulas for
$LL$, $U$, and $\imat$ are those shown in this section,
while the hazard and residuals all use the formulas of the prior section.
Although this is not perfectly consistent the numerical effect
on the residuals is minor, and it does not appear to affect
their utility.  
S uses the calculations of this section by default.

The robust variance for a Cox model is defined as $D'D$ where the 
$n \times p$ dfbeta matrix $D$ is based on the score residuals.  
Each row of $D$ represents the infinitesimal jackknife, the derivative
of $\bhat$ with respect to a change in the case weight for subject $i$.
It is fairly each to check this using a direct derivative,
$f(w_i + \epsilon) - f(w_i)/epsilon$ where $f$ is the vector of
coefficients from a fit of the Cox model
with the chosen weight for subject $i$ ($w_i$ will be 1 for most data sets).
This shows that the Efron/Breslow chimera is less accurate than the S
code.  However, I have not seen any example where the effect on either
$D$ or the robust variance $D'D$ was large enough to have practical 
consequences.  Still, the numerical analyst in me prefers to avoid an
inferior approximation.

\subsection{Exact partial likelihood}
At the tied death time the exact partial likelihood will have a
single term.
The numerator is a product of the risk scores of the subjects with an
event, and
the denominator is a sum of such products, where the sum is over
all possible choices of two subjects from the four who were at risk
at the time.
(If there were 10 tied deaths from a pool of 60 available, the
sum would be over all $\left( 60 \atop 10 \right)$ subsets,
a truly formidable computation!)
In our case, three of the four subjects at risk at time 6 have a risk score
of $\exp(0x)=1$ and one a risk score of $r$,
and the sum has six terms $\{r,r,r,1,1,1\}$.
\begin{eqnarray*}
LL &=& \{\beta- \log(3r+3)\} + \{\beta - \log(3r+3)\}  + \{0-0\} \\
   &=& 2\{\beta - \log(3r+3)\}. \\ \\
U &=& \left(1-\frac{r}{r+1}\right) + \left(1-\frac{r}{r+1}\right) + (0-0)\\ 
  &=& \frac{2}{r+1}. \\ \\
\imat&=&  \frac{2r}{(r+1)^2}. \\
\end{eqnarray*}

The solution $U(\beta)=0$ corresponds to $r=\infty$, with a loglikelihood
that asymptotes to $-2\log(3)$.  
The Newton--Raphson iteration has increments of $(r+1)/r$, 
so $\bhat=$ 0, 2, 3.1, 4.2, 5.2, and so on.
A solution at $\bhat=15$ is hardly different in likelihood from the
true maximum, however, and most programs will stop iterating shortly
after reaching this point.  
The information matrix, which measures the curvature of the likelihood
function, rapidly goes to zero as $\beta$ grows.

   Both SAS and S use the Nelson--Aalen estimate of hazard after
fitting an exact model, so the formulae of Table \ref{tab:val1} apply.
All residuals at $\bhat=0$ are thus identical to those for a
Breslow approximation.
At $\bhat=\infty$ the martingale residuals are still well defined.
Subjects 1 to 3, those with a covariate of 1, experience a hazard of
$r/(3r+3)=1/3$ at time 1.  Subject 3 accumulates a hazard of 1/3 at
time 1 and a further hazard of 2 at time 6.
The remaining subjects are at an infinitely lower risk during days
1 to 6 and accumulate no hazard then, with subject 6 being credited with
1 unit of hazard at the last event.
The residuals are thus $1-1/3=2/3$, $0-1/3$, $1-7/3= -4/3$, $1-0$, 0, and 0, 
respectively, for the six subjects.

Values for the score and Schoenfeld residuals can be
derived similarly as the limit as $r\rightarrow \infty$ of the formulae
in Section \ref{sect:valbreslow}.

\section{Test data 2}
This data set also has a single covariate, but in this case 
a (start, stop] style of input is employed.
Table \ref{tab:val2} shows the data sorted by the end time of the risk
intervals.  The columns for $\xbar$ and hazard are the values
at the event times; events occur at the end of 
each interval for which status = 1.
\begin{table}\centering
\begin{tabular}{ccc|ccc}
&&&Number& \\
Time&Status&$x$&at Risk& $\xbar$& $d\lhat$ \\ \hline 
(1,2] &1 &1 &2 &$r/(r+1)$    & $1/(r+1)$\\
(2,3] &1 &0 &3 & $r/(r+2)$   & $1/(r+2)$ \\
(5,6] &1 &0 &5 & $3r/(3r+2)$ & $1/(3r+2)$ \\
(2,7] &1 &1 &4 & $3r/(3r+1)$ & $1/(3r+1)$ \\
(1,8] &1 &0 &4 & $3r/(3r+1)$ & $1/(3r+1)$ \\
(7,9] &1 &1 &5 & $3r/(3r+2)$ & $2/(3r+2)$ \\
(3,9] &1 &1 && \\
(4,9] &0 &1 && \\
(8,14]&0 &0 &2& 0&0 \\
(8,17]&0 &0 &1& 0&0 \\
\end{tabular}
\caption{Test data 2}
\label{tab:val2}
\end{table}

\subsection{Breslow approximation}
For the Breslow approximation we have
\begin{eqnarray*}
LL &=&   \log\left(\frac{r}{r+1}\right)
	+\log\left(\frac{1}{r+2}\right)
	+\log\left(\frac{1}{3r+2}\right) +\\
    &&	\log\left(\frac{r}{3r+1}\right)
	+\log\left(\frac{1}{3r+1}\right)
	+2\log\left(\frac{r}{3r+2}\right)  \\
   &=& 4\beta - \log(r+1) - \log(r+3)- 3\log(3r+2) -2\log(3r+1). \\ \\
U &=& \left(1-\frac{r}{r+1}\right) + \left(0-\frac{r}{r+2}\right) 
	+ \left(0-\frac{3r}{3r+2}\right) + \\
    &&	 \left(1-\frac{3r}{3r+1}\right) + 
	 \left(0-\frac{3r}{3r+1}\right) + 
	 2\left(1-\frac{3r}{3r+2}\right) \\
  &=& 4-\frac{63s^4 + 201s^3 + 184s^2 + 48s}{9s^4 + 36s^3 +47s^2 + 24s +4}.
		\\ \\
\imat &=&  \frac{r}{(r+1)^2} + \frac{2r}{(r+2)^2} +
   	\frac{6r}{(3r+2)^2} + \frac{3r}{(3r+1)^2} \\
      && \frac{3r}{(3r+1)^2} + \frac{12r}{(3r+2)^2} . \\
\end{eqnarray*}

The solution is at $U(\bhat)=0$ or $r \approx .9189477$;
$\bhat = \log(r) \approx -.084529$.
Then
$$
\begin{array}{ll}
 LL(0)=-9.392662 & LL(\bhat)=-9.387015 \\
 U(0) = -2/15 & U(\bhat) = 0 \\
 \imat(0) = 2821/1800  & \imat(\bhat) = 1.586935.
\end{array}
$$

\def\haz{\hat \lambda}
The martingale residuals are (status--cumulative hazard) 
or $O-E = \delta_i - \int Y_i(s) r_i d\lhat(s)$.
Let $\haz_1, \ldots, \haz_6$ be the six increments to the cumulative
hazard listed in Table \ref{tab:val2}.  Then the cumulative hazards and
martingale residuals for the subjects are as follows.
\begin{center}
\begin{tabular}{c|lrr}
Subject &$\Lambda_0$ & $\Mhat(0)$ & $\Mhat(\bhat)$ \\ \hline
1& $r\haz_1$ 	& 1--30/60  &   0.521119 \\
2& $\haz_2 $	& 1--20/60   &  0.657411\\
3& $\haz_3 $ 	& 1--12/60   &  0.789777 \\
4& $r(\haz_2 + \haz_3 + \haz_4)$& 1--47/60 & 0.247388\\
5& $\haz_1+\haz_2+\haz_3+\haz_4+\haz_5$& 1--92/60 & --0.606293\\
6 &$r*(\haz_5 + \haz_6) $& 1--39/60 & 0.369025 \\
7& $r*(\haz_3+\haz_4+\haz_5+ \haz_6)$& 1--66/60&--0.068766 \\
8& $r*(\haz_3+\haz_4+\haz_5+ \haz_6)$& 0--66/60&--1.068766 \\
9& $\haz_6$ 	& 0--24/60 &  --0.420447 \\  
10& $\haz_6$ 	& 0--24/60 &  --0.420447   
\end{tabular}
\end{center}

The score and Schoenfeld residuals can be laid out in a tabular fashion.
Each entry in the table is the value of $\{x_i - \xbar(t_j)\} d\Mhat_i(t_j)$
for subject $i$ and event time $t_j$.  
The row sums of the table are the score residuals for the subject; 
the column sums are the Schoenfeld residuals at each event time.
Below is the table for $\beta= \log(2)$ ($r=2$).  
This is a slightly more stringent test than the table for $\beta=0$,
since in this latter case a program could be missing a factor of 
$r = \exp(\beta)=1$ and give the correct answer.
However, the results are much more compact than those for $\bhat$,
since the solutions are exact fractions.

%\newcommand{\pf}[2]{$\left(\frac{#1}{#2}\right)$}   %positive fraction
%\newcommand{\nf}[2]{$\left(\frac{-#1}{#2}\right)$}  %negative fraction
\newcommand{\pf}[2]{$\phantom{-}\frac{#1}{#2}$}   %positive fraction
\newcommand{\nf}[2]{$-\frac{#1}{#2}$}  %negative fraction
\renewcommand{\arraystretch}{1.5}

\begin{center}
\begin{tabular}{c|cccccc|c}
& \multicolumn{6}{c|}{Event Time} & Score\\
Id&2&3&6&7&8&9& Resid \\ \hline
1&\pf{1}{9} &&&&&&  $\phantom{-}\frac{1}{9}$ \\
2&&\nf{3}{8} &&&&& $-\frac{3}{8}$ \\
3&&&\nf{21}{32}&&&& $-\frac{21}{32}$ \\
4&&\nf{1}{4} & \nf{1}{16} & \pf{5}{49} &&&
		$-\frac{165}{784} $\\
5&\pf{2}{9}& \pf{1}{8} & \pf{3}{32} &
		\pf{6}{49} & \nf{36}{49}&& 
		$-\frac{2417}{14112} $\\
6&&&&&  \nf{2}{49} & \pf{1}{8}  & 
		$\phantom{-}\frac{33}{392}$ \\
7&&&  \nf{1}{16} &  \nf{2}{49}&  \nf{2}{49} &
	\pf{1}{8}  & $-\frac{15}{784}$ \\
8&&&  \nf{1}{16} &  \nf{2}{49}&  \nf{2}{49} &
	 \nf{1}{8} & $-\frac{211}{784}$ \\
9&&&&&& \pf{3}{16}  &$ \phantom{-}\frac{3}{16}$ \\
10&&&&&& \pf{3}{16}  & $\phantom{-}\frac{3}{16}$ \\ \hline
& \pf{1}{3} &\nf{1}{2}  & \nf{3}{4}  & 
	    \pf{1}{7} & \nf{6}{7}  & \pf{1}{2}  &  \nf{95}{84} \\

&&&&&&& \\
& $\frac{1}{r+1}$ & $\frac{-r}{r+2}$ & $\frac{-3r}{r+2}$ & $\frac{1}{3r+1}$ &
	$\frac{3r}{3r+1}$ & $\frac{4}{3r+2}$
\end{tabular}
\end{center}
Both the Schoenfeld and score residuals sum to the score statistic $U(\beta)$.
As discussed further above, programs will return two Schoenfeld residuals
at time 7, one for each subject who had an event at that time.

\subsection{Efron approximation}
This example has only one tied death time, so only the term(s) for the
event at time 9 change.  The main quantities at that time point are as follows.
\begin{center}
\begin{tabular}{r|cc}
&Breslow & Efron \\ \hline
$LL$ & $2\log\left(\frac{r}{3r+2}\right)$ &
       $\log\left(\frac{r}{3r+2}\right) + \log\left(\frac{r}{2r+2}\right)$ \\
$U$ &  $\frac{2}{3r+2}$& $\frac{1}{3r+2} + \frac{1}{2r+2}$ \\
$\imat$& $2\frac{6r}{(3r+2)^2} $ &$\frac{6r}{(3r+2)^2} + \frac{4r}{(2r+2)^2}$\\
$d\lhat$ & $\frac{2}{3r+2} $ &  $\frac{1}{3r+2} + \frac{1}{2r+2}$ 
\end{tabular}
\end{center}
\renewcommand{\arraystretch}{1}



\section{Test data 3}
This is very similar to test data 1, but with the addition of case
weights.
There are 9 observations, $x$ is a 0/1/2 covariate, and weights range from
1 to 4.
As before, let $r = \exp(\beta)$ be the risk score for a subject with $x=1$.
Table \ref{tab:val3} shows the data set along with the mean and
increment to the hazard at each point.

\begin{table} \centering
\begin{tabular}{cccc|cc}
Time&Status&$X$ & Wt& $\xbar(t)$ & $d\lhat_0(t)$ \\ \hline
1& 1& 2 & 1& $(2r^2+11r) d\lhat_0 =\xbar_1$ & $1/(r^2 + 11r +7)$ \\
1& 0& 0 & 2&& \\
2& 1& 1 & 3& $11r/(11r+5) = \xbar_2$ & $10/(11r+5)$ \\
2& 1& 1 & 4&& \\
2& 1& 0 & 3&& \\
2& 0& 1 & 2&& \\
3& 0& 0 & 1&& \\
4& 1& 1 & 2&   $2r/(2r+1)= \xbar_3$ & $ 2/(2r+1)$ \\
5& 0& 0 & 1 & \\
\end{tabular}
\caption{Test data 3}
\label{tab:val3}
\end{table}

\subsection{Breslow estimates}
The likelihood is now a product of terms, one for each death, of
the form
$$
    \left( \frac{e^{X_i \beta}}{\sum_j Y_j(t_i) w_j e^{X_j \beta}} 
    \right) ^{w_i}
$$
leading to a log-likelihood very like equation \ref{eq:loglik}
\begin{equation}
l(\beta) = \sum_{i=1}^n \int_0^\infty \left[
	  X_i(t)\beta - \log \left( \sum_j Y_j(t)w_j r_j(t) \right)
	 \right] w_i dN_i(t)
    \label{eq:wtloglik}
\end{equation}

For integer weights, this gives the same results as would
be obtained by replicating each observation the specified number of times,
which was in fact one motivation for the definition.
The definitions for the score vector $U$ and information matrix $\imat$
simply replace the mean and variance with weighted versions of the same.
Let $l(\beta,w)$ be the logliklihood when all the observations are given
a common case weight of $w$; 
it is easy to prove that $l(\beta,w) = wl(\beta,1) - d\log(w)$ where $d$ is
the number of events.
One consequence of this is that the log-likelihood can be positive when many
of the weights are $<1$, which
sometimes occurs in survey sampling applications.  (This can be a big
surprise the first time one encounters the situation.)
\begin{eqnarray*}
LL &=& \{2\beta - \log(r^2 + 11r +7)\} + 3\{\beta - \log(11r+5)\} \\
   &&	+ 4\{\beta - \log(11r+5)\}  +3\{0 - \log(11r+5)\}  \\
   &&	+  2\{\beta - \log(2r+1) \} \\
   &=& 11\beta - \log(r^2 + 11r +7) -10\log(11r+5) - 2\log(2r+1) \\\\
U  &=& (2- \xbar_1) + 3(0-\xbar_2) + 4(1-\xbar_2) + 3(1-\xbar_2) + 
	2(1-\xbar_3) \\ 
   &=&  11 - (\xbar_1 + 10\xbar_2 + 2\xbar_3)  \\
I  &=& [(4r^2+11r)/(r^2+11r+7)- \xbar_1^2] + 10(\xbar_2 - \xbar_2^2) 
	+ 2(\xbar_3 - \xbar_3^2) \\
\end{eqnarray*}

 The solution corresponds to $U(\beta)=0$, which is the solution point of 
the polynomial $66r^4 + 425r^3 - 771r^2 - 1257r -385 =0$,
or $\bhat \approx \log(2.3621151) = 0.8595574$.
Then
$$
\begin{array}{ll}
 LL(0)=-32.86775 & LL(\bhat)=-32.02105 \\
 U(0) = 2.107456 & U(\bhat) = 0 \\
 \imat(0) = 2.914212 & \imat(\bhat) = 1.966563 
\end{array}
$$

When $\beta=0$, the three unique values for $\xbar$
at $t=1$, 2, and 4
are 13/19, 11/16 and 2/3,
respectively, and the increments to the cumulative hazard are
1/19, 10/16 = 5/8, and 2/3, see table \ref{tab:val3}.
The martingale and score residuals at $\beta=0$ are
$\bhat$ of
\begin{center}
\begin{tabular}{cc|lr}
Id &Time& \multicolumn{1}{c}{$M(0)$} &\multicolumn{1}{c}{$M(\bhat)$} \\ \hline
A&1 & $1-1/19 = 18/19             $&0.85531\\
B&1 & $0-1/19 = -1/19             $&-0.02593\\
C&2 & $1-(1/19 + 5/8)= 49/152     $&0.17636 \\
D&2 & $1-(1/19 + 5/8)= 49/152     $&0.17636\\
E&2 & $1-(1/19 + 5/8)= 49/152     $&0.65131\\
F&2 & $0-(1/19 + 5/8)= -103/152     $&-0.82364\\
G&3 & $0-(1/19 + 5/8)=  -103/152     $&-0.34869\\
H&4 & $1-(1/19 + 5/8 +2/3)= -157/456 $&-0.64894\\
I&5 & $0-(1/19 + 5/8 +2/3)= -613/456 $&-0.69808\\
\end{tabular}
\end{center}
Score residuals at $\beta=0$ are
\begin{center}
\begin{tabular}{cc|r}
Id &Time& Score \\ \hline
A&1 &$(2-13/19)(1-1/19)$\\
B&1 &$(0-13/19)(0-1/19)$\\
C&2 &$ (1-13/19)(0-1/19) + (1-11/16)(1-5/8) $ \\
D&2 &$(1-13/19)(0-1/19) + (1-11/16)(1-5/8)$ \\
E&2 &$(0-13/19)(0-1/19) + (0-11/16)(1-5/8)$ \\
F&2 &$(1-13/19)(0-1/19) + (1-11/16)(0-5/8)$ \\
G&3 &$(1-13/19)(0-1/19) + (0-11/16)(0-5/8)$ \\
H&4 &$(1-13/19)(0-1/19) + (1-11/16)(0-5/8) $ \\
      &&                $  + (1-2/3)(1-2/3)$ \\
I&5 &$(1-13/19)(0-1/19) + (1-11/16)(0-5/8)$  \\
      &	&$+ (0-2/3)(0-2/3) $ \\
\end{tabular}
\end{center}

SAS returns the unweighted residuals as given above;
it is the weighted sum of residuals that totals zero, $\sum w_i \Mhat_i=0$,
likewise for the score and Schoenfeld residuals evaluated at $\bhat$.
S also returns unweighted residuals by default, with an option to
return the weighted version.
Whether the weighted or the unweighted form is more useful depends on the
intended application, neither is more ``correct'' than the other.
S does differ for the dfbeta residuals, for which the default is
to return weighted values.  For the third observation in this data set,
for instance, the unweighted dfbeta is an approximation to the change in
$\bhat$ that will occur if the case weight is changed from 2 to 3,
corresponding to deletion of one of the three ``subjects'' that this
observation represents, and the weighted form approximates a change in 
the case weight from 0 to 3, i.e., deletion of the entire observation.

 The increments of the Nelson-Aalen estimate of the hazard are shown
in the rightmost column of table \ref{tab:val3}.
The hazard estimate for a hypothetical subject with covariate $X^\dagger$ is
$\Lambda_i(t) = \exp(X^\dagger \beta) \Lambda_0(t)$ and the survival
estimate is $S_i(t)= \exp(-\Lambda_i(t))$.
The two term of the variance, for $X^\dagger=0$, are Term1 + $d'Vd$:
\begin{center}
\begin{tabular}{c|l}
Time & Term 1  \\ \hline
1& $1/(r^2 + 11r+7)^2$ \\
2& $1/(r^2 + 11r+7)^2 + 10/(11r+5)^2$ \\
4& $1/(r^2 + 11r+7)^2 + 10/(11r+5)^2 + 2/(2r+1)^2$ \\ 
\multicolumn{2}{c}{} \\
Time  & $d$ \\ \hline
1&  $(2r^2+11r)/(r^2+11r+7)^2$ \\
2&  $(2r^2+11r)/(r^2+11r+7)^2 + 110r/(11r+5)^2$ \\
4&  $(2r^2+11r)/(r^2+11r+7)^2 + 110r/(11r+5)^2 + 4r/(2r+1)^2$
\end{tabular}
\end{center}

 For $\beta=\log(2)$ and  $X^\dagger =0$, 
where $k\equiv$  the variance of
$\bhat$ = 1/2.153895 this reduces to
\begin{center}
\begin{tabular}{c|ll}
     Time  &   \multicolumn{2}{c}{Variance}\\ \hline 
       1   &      1/1089                &+ $k(30/1089)^2$ \\
       2   &     (1/1089+ 10/729)       &+ $k(30/1089+ 220/729)^2   $ \\
       4   &     (1/1089+ 10/729 + 2/25)&+ $k(30/1089+ 220/729 + 8/25)^2$\\
\end{tabular}
\end{center}
giving numeric values of 0.0012706, 0.0649885, and 0.2903805, 
respectively.

\subsection{Efron approximation}
For the Efron approximation the combination of tied times and case weights
can be approached in at least two ways.  
One is to treat the case weights as replication counts.  There are then
10 tied deaths at time 2 in the data above, and the Efron approximation
involves 10 different denominator terms.  
Let $a= 7r+3$, the sum of risk scores for the 3 observations with an event
at time 2 and $b=4r+2$, the sum of risk scores for the other subjects at
risk at time 2.
For the replication approach, the loglikelihood is 
\begin{eqnarray*}
LL &=& \{2\beta - \log(r^2 + 11r +7)\} + \\ 
   &&	\{7\beta - \log(a+b) - \log(.9a+b) - \ldots - \log(.1a+b) \} + \\
   &&	 \{2\beta - \log(2r+1) - \log(r+1)\}. 
\end{eqnarray*}
A test program can be created by comparing results from the weighted
data set (9 observations) to the unweighted replicated data set (19
observations).
This is the approach taken by SAS \texttt{phreg} using the \texttt{freq}
statement.  It's advantage is that the appropriate
result for all of the weighted computations is perfectly clear,
the disadvantage is that only integer case weights are supported.
(A second advantage is that I did not need to create another
algebraic derivation for my test suite.)

A second approach, used in S, allows for non-integer weights.
SAS also has weighted estimates, but I am not familiar with their algorithms.
The data is considered to be 3 tied observations, and the
log-likelihood at time 2 is the sum of 3 weighted terms.
The first term of the three is one of
\begin{eqnarray*}
	&& 3 [\beta - \log(a+b)]	\\
	&& 4 [\beta - \log(a+b)]     \\
 &{\rm or}&3 [0 - \log(a+b)],
\end{eqnarray*}
depending on whether the event for observation C, D or E 
actually happened first
(had we observed the time scale more exactly); 
the leading multiplier of 3, 4  or
3 is the case weight.
The second term is one of 6 possiblities
\begin{eqnarray*}
	& 4 [\beta - \log(4r+3+b)]  & CDE \\
        & 3 [\beta - \log(4r+3+b)]  & CED \\	
	& 3 [0 -     \log(3r+3+b)]  & DCE\\
	& 3 [0     - \log(3r+3+b)]  & DEC \\
        & 3 [\beta - \log(7r+0+b)]  & ECD\\
{\rm or}	& 4 [\beta - \log(7r+0+b)]  & EDC
\end{eqnarray*}
The first choice corresponds to an event order of observation C then D
(subject D has the event, with D and E still at risk), etc.
For a weighted Efron approximation first replace each term by its average,
just as in the unweighted case.
The first terms ends up as $(7/3)\beta - (10/3)\log(a+b)$,
the second as $(7/3)\beta - 20/6\log(2a/3 + b)$, and the third
as $(7/3) \beta - 10/3\log(a/3 +b)$. 
This replaces the interior of the log function with its average, and the
multiplier of the log with the average weight of 10/3.

The final log-likelihood and score statistic are
\begin{eqnarray*}
LL &=&  \{2\beta - \log(r^2 + 11r +7)\} \\ &&  + 
	\{7\beta - (10/3)[\log(a+b) + \log(2a/3 +b) + \log(a/3+b)] \} \\&& +
	 2\{\beta - \log(2r+1) \} \\ \\
U  &=& (2- \xbar_1) + 2(1-\xbar_3) \\
   && + 7 - (10/3)[\xbar_2 + 26r/(26r+12) + 19r/(19r+9)] \\
   &=& 11 -(\xbar_1 + (10/3)(\xbar_2 + \xbar_{2b} +\xbar_{2c}) 
		+ 2\xbar_3)\\ \\
I  &=& [(4s^2+11s)/(s^2+11s+7)- \xbar_1^2] \\ 
   &&+ (10/3)[ (\xbar_2- \xbar_2^2) + (\xbar_{2b}- \xbar_{2b}^2)
			+ (\xbar_{2b}- \xbar_{2b}^2) \\
   &&+2(\xbar_3 - \xbar_3^2) \\
\end{eqnarray*}
The solution is at $\beta=.87260425$, and 
$$
\begin{array}{ll}
 LL(0)=-30.29218 & LL(\bhat)=-29.41678 \\
 U(0) = 2.148183 & U(\bhat) = 0 \\
 \imat(0) = 2.929182 & \imat(\bhat) = 1.969447 \,.
\end{array}
$$

The hazard increment and mean at times 1 and 4 are identical to those
for the Breslow approximation, as shown in table \ref{tab:val3}.
At time 2, the number at risk for the first, second and third portions
of the hazard increment
are $n_1= 11r+5$, $n_2= (2/3)(7r+3) + 4r+2 = (26r+12)/3$, and
$n_3=(1/3)(7r+3) + 4r+2 = (19r+9)/3$.
Subjects F--I experience the full  hazard at time 2 
of $(10/3)(1/n_1 + 1/n_2 + 1/n_3)$,
subjects B--D experience $(10/3)(1/n_1 + 2/3n_2 + 1/3n_3)$.
Thus, at $\beta=0$ the martingale residuals are
\begin{center}
\begin{tabular}{cc|ll}
Id& Time & \multicolumn{1}{c}{$\Mhat(0)$} \\ \hline
A & 1 & 1 - 1/19 & = 18/19 \\
B & 1 & 0 - 1/19 & = -1/19 \\
C & 2 & 1 - (1/19 + 10/48 + 20/114 + 10/84)& =473/1064 \\
D & 2 & 1 - (1/19 + 10/48 + 20/114 + 10/84)& =473/1064 \\
E & 2 & 1 - (1/19 + 10/48 + 20/114 + 10/84) &=473/1064 \\
F & 2 & 0 - (1/19 + 10/48 + 10/38\phantom{4}  + 10/28)& =-2813/3192 \\
G & 3 & 0 - (1/19 + 10/48 + 10/38\phantom{4}  + 10/28) &=-2813/3192 \\
H & 4 & 1 - (1/19 + 10/48 + 10/38\phantom{4}  + 10/28 + 2/3) &=-1749/3192 \\
I & 5 & 0 - (1/19 + 10/48 + 10/38\phantom{4}  + 10/28 + 2/3) &=-4941/3192
\end{tabular}
\end{center}

The hazard estimate for a hypothetical subject with covariate $X^\dagger$ is
$\Lambda_i(t) = \exp(X^\dagger \beta) \Lambda_0(t)$, 
$\Lambda_0$ has increments of $1/(r^2 + 11r +7$,  
$(10/3)(1/n_1 + 1/n_2 + 1/n_3)$ and $2/(2r+1)$.  
This increment at time 2 is a little larger than the Breslow jump of
$10/d1$.
The first term of the variance will have an increment of
$[\exp((X^\dagger \beta)(]^2 (10/3)(1/n_1^2 + 1/n_2^2 + 1/n_3^2)$ at time 2.
The increment to the cumulative distance from the center $d$ will be
\begin{eqnarray*}
  && \left[X^\dagger - \frac{11r}{11r+5} \right] \frac{10}{3 n_1} \\
  &+& \left[X^\dagger -\frac{(2/3)7r + 4r}{n2} \right] (10/3)(1/n_2) \\
  &+& \left[X^\dagger -\frac{(1/3)7r + 4r}{n2} \right] (10/3)(1/n_3) 
\end{eqnarray*}

For $X^\dagger = 1$ and $\beta=\pi/3$ we get cumulative hazard and variance 
below.  We have $r= e^\pi/3$, 

\begin{tabular}{ll|llr}
\multicolumn{2}{c}{$\Lambda$} & \multicolumn{3}{c}{Variance} \\ \hline
$e/(r^2+ 11r+7)$ &  0.03272832 & $e^2/(r^2+ 11r+7)^2$ & 
\end{tabular}

\section{Test data 4}
This is an extension of test data set 3, but with 3 covariates.
Let $r_i = \exp(X_\beta)$ be the risk score for each subject, $\beta$
unspecified.
Table \ref{tab:val4} shows the data set along with the mean and
increment to the hazard at each point.

\begin{table} \centering
\begin{tabular}{ccccccc|cc}
Id&Time&Status&$x_1$ &$ x_2$& $x_3$ &Wt& Denominator& $\xbar_2$\\
    \hline
1&10& 1& 0& 2 &5 & 1& $d_1= r_1 + 2r_2 + d_2 + d_3$ & 
	  $(2r_1 + 3r_3+ 4r_4 + 2r_6+ 2r_8)/d_1$ \\
2&10& 0& 0& 0 &2&  2&& \\
3&20& 1& 1& 1 &3&  3& $d_2= 3r_3 + 4r_4 +  3r_5 +$ &
	          $(3r_3+ 4r_4 + 2r_6+ 2r_8)/d_2$   \\                
4&20& 1& 1& 1 &6&  4& \qquad $2r_6 + r_7 + d_3$& \\
5&20& 1& 0& 0 &4&  3&& \\
6&20& 0& 0& 1 &3&  2&& \\
7&30& 0& 1& 0 &1&  1&& \\
8&40& 1& 1& 1 &3&  2&   $d_3= 2r_8 + r_9$ & $2r_4/d_3$ \\
9&50& 0& 1& 0 &1&  1 & \\
\end{tabular}
\caption{Test data 4}
\label{tab:val4}
\end{table}

At $\beta=0$ we have $r=1$ and

\end{document}